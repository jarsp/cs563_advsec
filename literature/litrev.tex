% arara: pdflatex
% arara: biber
% arara: pdflatex
% arara: pdflatex
\documentclass[11pt]{article}

\usepackage{biblatex}
\addbibresource{bibliography.bib}

\usepackage{geometry}

\begin{document}

\section{AEG via Symbolic Methods}

Various variants of AEG (Automatic Exploit Generation) have been studied, most
often by Brumley's team at CMU. Nonexhaustively:

\begin{itemize}
\item
    Automatic Exploit Generation \cite{AEG}
    \begin{itemize}
    \item
        Demonstrates end-to-end system for generating exploits from the source
        of an application, ran on various small-medium applications successfully
    \item
        Entirely based on (dynamic) symbolic (and concolic) execution,
        leveraging KLEE
    \item
        Preconditioned symbolic execution, i.e. using heuristics to prune
        symbolic branches, to mitigate path explosion
    \item
        Environment modelling to handle syscalls and other interactions
    \item
        They have a short updated article in 2014 \cite{AEG14} that presents
        their binary AEG tool Mayhem. Report also includes a short history of
        AEG
    \end{itemize}

\item
    Automatic patch-Based Exploit Generation is Possible: Techniques and
    Implications \cite{APEG}
    \begin{itemize}
    \item
        Generating exploit for binaries that have been patched, by examining
        differences
    \item
        Assumes that patches for (input sanitization) vulnerabilities typically
        add some new check, so generates candidate exploits for the unpatched
        version that fail these checks
    \item
        Handles path explosion by doing dynamic analysis up to some instruction
        $i$ on the path to the new check, and then does static analysis
        (i.e. model checking) for the remainder of the path to the new check.
        Tries to find a combination of the two solutions that are compatible
    \end{itemize}

\item
    CRAX: Software Crash Analysis for Automatic Exploit Generation by Modeling
    Attacks as Symbolic Continuations \cite{CRAX}
    \begin{itemize}
    \item
        Uses crashing executions to guide concolic execution to find bugs
        (primarily control hijack attacks) in binaries
    \item
        Concretization of irrelevant functions (standard practice in dynamic
        symbolic execution/concolic execution)
    \item
        Uses S$^2$E for system simulation
    \item
        Runs on large applications (approx. 1-2M LOC)
    \end{itemize}
\end{itemize}

\section{AEG via Hybrid Fuzzing/Symbolic Execution}

More recently, esp. wrt. CGC \cite{CGC}, combinations of fuzzing and symbolic
execution techniques have become more popular. Certainly worth looking at all
the CGC finalists. CGC also involved automated patching of vulnerabilities,
which is something interesting to look at as well.

\begin{itemize}
\item
    Cyber Grand Shellphish \cite{CGS}
    \begin{itemize}
    \item
        A report by the CGC Shellphish team in Phrack
    \item
        Bunch of explanation of the contest format as well as their approach,
        and a number of other ideas
    \item
        TODO: Finish reading, probably has lots of references as well
    \end{itemize}

\item
    Driller: Augmenting Fuzzing Through Selective Symbolic Execution
    \cite{DRILLER}
    \begin{itemize}
    \item
        Important paper that improved the state-of-the-art by augmenting
        fuzzing with symbolic execution in a novel manner
    \item
        Key idea: there are broadly two types of inputs to conditional
        branches, namely 'general input' of which many different inputs could
        work, and 'specific input', of which only a tiny fraction of inputs
        pass the check. Specific input checks logically separate the application
        into compartments. Exploration within a compartment is easily handled
        by a fuzzer, while pushing exploration between compartments is best
        handled by symbolic execution. This helps mitigate path explosion in
        symbolic execution as well as we are usually looking just one part of
        the application, and lets us have the benefits of fuzzing.
    \item
        The basic idea of combining the two has been done before, but usually
        in the form of fuzz, then symbolically execute, and negate the
        accumulated precondition to gain maximum code coverage. However it lacks
        the key insight above, and ends up wasting time by exploring branches
        within a single compartment using symbolic execution.
    \item
        This paper was the basis for the Shellphish team, they compare it to
        other techniques as well, a lot of references to look at. Have to look
        a bit more at this paper as well.
    \end{itemize}
\end{itemize}

\section{Background and Theory}

There's a fair amount of background theory that goes into these things, but the
basic ideas are fairly simple. Here are some topics to look out for, feel free
to add more:

\begin{itemize}
\item
    Basic first-order logic concepts, e.g. models, satisfiability, entailment
    provability, soundness, completeness, etc.. Most texts on first order logic
    will do. \cite{COC} is a fairly good one and also talks a little bit about
    decision procedures which are used in SMT solvers, although that is a little
    bit less relevant for us (probably)
\item
    Satisfiability Modulo Theories (SMT) solvers. The canonical example of an
    SMT solver is Microsoft's Z3 \cite{Z3}
\item
    (Dynamic) Symbolic Execution, and Concolic Execution. There are many
    resources on this with varying degrees of clarity, here is one possibility
    \cite{SymEx}
\item
    "Static" Symbolic Execution, a.k.a. (Bounded) Model Checking. Once again
    there are many resources on this. There are many different ways to do
    model checking, \cite{LICS} looks at some of the ideas, though probably
    only some of the key ideas are relevant
\end{itemize}

\printbibheading
\printbibliography

\end{document}
